\documentclass{wg21}

\input{glyphtounicode}
\pdfgentounicode=1

\usepackage{xcolor}
\usepackage{soul}
\usepackage{ulem}
\usepackage{fullpage}
\usepackage{parskip}
\usepackage{csquotes}
\usepackage{relsize}      % provide relative font size changes
\usepackage{textcomp}     % provide \text{l,r}angle
\usepackage{mathrsfs}     % mathscr font
\usepackage{microtype}

\title{Make \texttt{std::random\_device} Scrutable}
\docnumber{PXXXXR0}
\audience{LEWG}
\author{Martin Hořeňovský}{martin.horenovsky@gmail.com}

\usepackage[inner=2cm,outer=2cm]{geometry}

\begin{document}
\maketitle

%\begin{flushright}
%\hfill \break
%\hfill \break
%\textit{Some tagline?}
%\end{flushright}

% random device is non-deterministic until it isn't
% https://eel.is/c++draft/rand#device

\hypertarget{introduction}{%
    \section{Introduction}\label{introduction}}

The C++ standard library provides a set of utilities to generate
pseudo-random numbers with different properties. It also provides
\tcode{std::random_device} for users who desire nondeterministic
randomness.

However, implementations are also allowed to employ a deterministic
pseudo-random number generator if providing nondeterministic randomness
is impossible. This allows implementations to provide
\tcode{std::random_device} even on any possible platform, but there
is a problem with this: the current interface does not provide a good
way of checking whether the implementation provides non-deterministic
random numbers.


\hypertarget{motivation}{%
    \section{Motivation}\label{motivation}}

The standard already provides a way, specifically the
\tcode{std::random_device::entropy()} member method, that the user should
be able to use to find out whether \tcode{random_device} provides
nondeterministic random number. In practice, this approach has two large
flaws:

\begin{itemize}
    \item This information is only available at runtime. In practice,
          an implementation already knows whether it provides
          nondeterministic during compilation, and we should expose this
          information to the users.
    \item Implementations do not always implement this function usefully.
          For example, older versions of libstdc++ used to always return 0.
          Current versions use Linux-specific syscall to query the
          available entropy, and on Windows they return 0 even if they use
          OS-provided secure random facilities. Similarly, libc++ only ever
          returns 0, even though it only uses platform-provided secure
          random facilities.
\end{itemize}

The second flaw is likely exacerbated by the fact that if an implementation
is using a platform-provided random facilities, it likely cannot provide
a real and meaningful entropy estimate.

\hypertarget{non-goals}{%
    \section{Non-goals}\label{non-goals}}

There are several other problems with \tcode{<random>}, and even
\texttt{std::random_device} as it is currently standardized. This paper
intentionally does not try to fix any of them.
This is because there are other papers that target those issues, and we
should not let fixing an issue get bogged down because of a lack of consensus
on other issues.


\hypertarget{proposed-solution}{%
    \section{Proposed solution}\label{proposed-solution}}

To address these issues, I propose to deprecate the \tcode{entropy}
member function and instead add \tcode{static constexpr bool is_predictable()}%
\footnote{Possible alternative names that could be used instead:
\tcode{is_nonpredictable}, \tcode{is_deterministic},
or \tcode{is_nondeterministic}}
function to \tcode{random_device}. This function returns \tcode{true} if
\mbox{\tcode{random_device}} is implemented in a way that makes its
output predictable, e.g. by using a PRNG and \tcode{false} when its output
is not predictable, e.g. when it is implemented in terms of OS-provided
cryptographically secure random number generation facilities.

The reason to deprecate \tcode{entropy} is that only a small niche in
the Linux community thinks of entropy in the OS providded CSRNG facilities
as depletable. Combined with the fact that 2 out of the big 3
implementations do not return meaningful results on all platforms, and
that the users are more interested in simple boolean query of ``will
the output be unpredictable'', \tcode{entropy} is not useful in any way.

Even for the niche case where a user on the Linux platform cares about
the estimated entropy in \tcode{/dev/urandom}, the interface has a TOCTOU
problem.


\hypertarget{constexpr-query-wording}{%
    \subsection{Proposed wording changes}\label{constexpr-query-wording}}


\begin{codeblock}
    class random_device {
        public:
        // types
        using result_type = unsigned int;
        
        // generator characteristics
        static constexpr result_type min() { return numeric_limits<result_type>::min(); }
        static constexpr result_type max() { return numeric_limits<result_type>::max(); }
        
        // constructors
        random_device() : random_device(implementation-defined) {}
        explicit random_device(const string& token);
        
        // generating functions
        result_type operator()();
        
        // property functions
        double entropy() const noexcept;
        @\added{static constexpr bool is_predictable();}@
        
        // no copy functions
        random_device(const random_device&) = delete;
        void operator=(const random_device&) = delete;
    };
\end{codeblock}

\begin{codeblock}
@\added{static constexpr bool is_predictable();}@
\end{codeblock}

\begin{itemdescr}
    \added{\returns{} \tcode{true} if default-constructed \tcode{random_device}
    uses a predictable random generator, such as Mersenne Twister.
    \tcode{false} otherwise.}
\end{itemdescr}



\hypertarget{alternative-approach}{%
    \section{Alternative approach}\label{alternative-approach}}

An alternative approach is to avoid deprecating \tcode{entropy()}
and instead clarifying that \tcode{entropy()} should only
return 0 for predictable implementations, such as those that rely on a
random number engine as defined in [rand.req.eng], and implementations
that use unpredictable implementations, such as using \tcode{rand_s}
provided by Windows, should return non-zero estimate.



\hypertarget{entropy-query-wording}{%
    \subsubsection{Proposed wording changes}\label{entropy-query-wording}}

[rand.device]:

\begin{itemdescr}
    If implementation limitations prevent generating nondeterministic
    random numbers, the implementation may employ a random number engine
    \added{as defined in [rand.req.eng]}.
\end{itemdescr}

\vskip 0.5cm


\begin{codeblock}
double entropy() const noexcept;
\end{codeblock}

\begin{itemdescr}
    \returns If the implementation employs a random number engine, returns 0.0. Otherwise, returns a \added{non-zero} entropy estimate for the random numbers returned by operator(), in the range min() to log2(max()+1).
\end{itemdescr}



\hypertarget{goals}{%
    \section{Goals}\label{goals}}

This paper has two goals, 
\begin{enumerate}
    \item Provide users of \tcode{random_device} with a way to query
    whether the output is predictable or not.
    \item Start a discussion on what should happen to
    \tcode{random_device::entropy}
\end{enumerate}


\end{document}
