\documentclass{wg21}

\input{glyphtounicode}
\pdfgentounicode=1

\usepackage{xcolor}
\usepackage{soul}
\usepackage{ulem}
\usepackage{fullpage}
\usepackage{parskip}
\usepackage{csquotes}
\usepackage{relsize}      % provide relative font size changes
\usepackage{textcomp}     % provide \text{l,r}angle
\usepackage{mathrsfs}     % mathscr font
\usepackage{microtype}

\title{Make Pseudo-random Numbers Portable}
\docnumber{DXXXXR0}
\audience{LEWG}
\author{Martin Hořeňovský}{martin.horenovsky@gmail.com}

\usepackage[inner=2cm,outer=2cm]{geometry}

\begin{document}
\maketitle

%\begin{flushright}
%\hfill \break
%\hfill \break
%\textit{Some tagline?}
%\end{flushright}

% URNGs are portable:
% http://eel.is/c++draft/rand#predef

\hypertarget{introduction}{%
    \section{Introduction}\label{introduction}}

The C++ standard library provides a set of utilities to generate
pseudo-random numbers in a range of user's choosing. This is done
by applying a type that represents the desired statistical distribution
(e.g. \texttt{uniform\_int\_distribution}) with a type conforming
to the Uniform Random Bit Generator (URBG) concept (e.g. \texttt{mt19937}).

This makes the standard-provided facilities highly customizable, but they
are rarely used because the generated numbers are not reproducible across
different platforms. The reason for this is that the output of
standard-provided distributions\footnote{The output of standard-provided
URBGs is portable due to a happy accident during the original
\texttt{<random>} standardization} is implementation-specified and thus
can (and does) differ across implementations.

This paper proposes changes that would allow users to use these facilities
for domains that require cross-platform reproducibility, such as testing,
scientific simulations, and games\footnote{See \nameref{app-a-uses} for
more use cases that require cross-platform reproducibility.}.


\hypertarget{motivation}{%
    \section{Motivation}\label{motivation}}

A straightforward ``Hello world'' of random number generation looks something
like this:

% https://godbolt.org/z/dB7H_2
\begin{codeblock}
#include <random>
#include <iostream>

int main() {
    std::mt19937 urbg;
    std::uniform_int_distribution<> dist(0, 100);
    std::cout << dist(urbg) << '\n';
}
\end{codeblock}

On my machine\footnote{MSVC bundled with VS 16.4.0}, this code always
prints out \texttt{53}. \href{https://godbolt.org/z/dB7H_2}{On other
popular platforms, the code prints \texttt{82}\footnote{libstdc++} and
\texttt{92}\footnote{libc++}}. This
difference in outputs makes the standard-provided \texttt{<random>}
facilities unusable in contexts where a cross-platform reproducibility
is required, such as procedural generation and physics simulations.

However, cross-platform reproducibility is useful even in cases
where it is not required, such as property-based tests. For property-based
testing, it is desirable for a fixed seed to generate the same
test case across different platforms, so that developers can share
relevant seeds between themselves.

Because of this, \texttt{<random>} is currently unusable in many
domains and applications, and if we want to change that, we need
to ensure that a deterministically seeded code, such as the snippet
above, generates the same numbers on all platforms.

\hypertarget{non-goals}{%
    \section{Non-goals}\label{non-goals}}

There are several other problems with \texttt{<random>} as it is currently
standardized, and this paper intentionally does not try to fix any of them.
This is because there are other papers that target those issues, and we
should not let fixing an issue get bogged down because of a lack of consensus
on other issues.


\hypertarget{possible-solutions}{%
    \section{Possible solutions}\label{possible-solutions}}

This section outlines possible solutions to the problem explained above,
together with their advantages and disadvantages as seen by the author of
this paper.

\hypertarget{standardize-implementation}{%
    \subsection{Standardize implementation for current distributions}\label{possible-solutions}}

One possibility (the one prefered by the author), is to mandate the specific
implementation of all distributions in \texttt{<random>}. This option
has two advantages, in that

\begin{itemize}
    \item it retroactively fixes broken code
    \item it does not introduce further maintenance burden on implementations of the standard library
\end{itemize}

They are also disadvantages at the same time, in that

\begin{itemize}
    \item it is a (potentially) breaking change
    \item it forbids implementations from providing the fastest possible code for their platform
\end{itemize}


The second disadvantage can be fixed by adding \texttt{fast_meow_distribution}s
to the standard.


\hypertarget{portable-overloads}{%
    \subsection{Provide \texttt{portable_meow_distributions}}\label{portable-overloads}}

Another possibility is to add \texttt{portable_meow_distributions}
to \texttt{<random>}. This option is essentially the inverse solution
of the first option, in that the old code is left as it was, and
a new set of types is provided for portability.

However, this option is less desirable one as it makes the simpler code
do the wrong thing, and is harder to teach.

\hypertarget{specific-algorithms}{%
    \subsection{Provide specific algorithms as distributions}\label{specific-algorithms}}

The last option is to keep the status quo for already standardized types,
but also provide specific algorithms under their own name, e.g. provide
an implementation of
\href{https://en.wikipedia.org/wiki/Marsaglia_polar_method}{Marsaglia polar method}
as \mbox{\texttt{std::marsaglia_polar_method_distribution}}.

This option has some compeling advantages, in that

\begin{itemize}
    \item it lets experts pick the right algorithm for their domain
    \item allows for future expansion with better portable algorithms as they are found
    \item does not break backward compatibility
\end{itemize}

However, this option is the least teachable option by far, and it is also
the least beginner friendly of all options presented in this paper. The
common C++ programmer wants to generate some reproducible normally
distributed numbers, without having to know the differences between
Marsaglia polar method, Box-Muller transform and the Ziggurat algorithm.

It also does not absolve the standard from prescribing at least parts
of the implementation of specific algorithms. As an example, the already
mentioned Marsaglia polar method generates 2 normally distributed floating
point numbers in single run. The order in which these 2 numbers are returned
would need to be specified in the standard to avoid differences in behaviour
between different implementations\footnote{This difference already exists
in the wild, as both MSVC and libstdc++ use this algorithm for their
implementation of \texttt{std::normal_distribution}, but return
the generated numbers in different order.}.


\hypertarget{goals}{%
    \section{Goals}\label{goals}}

The goal of this paper is to start discussion and find a consensus on 3 questions:

\begin{enumerate}
    \item Should the standard provide portable distributions?
    \item Which of the three approaches should standard take?
    \item How should the standard deal with distributions on floating point numbers?\footnote{Remember that the standard does not even specify whether floating point numbers adhere to IEEE-754.}
\end{enumerate}


\newpage

\appendix

\hypertarget{app-a-uses}{%
    \section{Appendix A: Uses for repeatable distributions}\label{app-a-uses}}

To come up with a more comprehensive list of problems where people need
repeatable random number generation,
\href{https://www.reddit.com/r/cpp/comments/e9mft4/when_did_you_want_to_use_random_but_couldnt/}
{I opened up a Reddit thread}.

This is a short list of the answers:

\begin{itemize}
    \item Fuzzing in general
    \item Property based testing
    \item Benchmarking of data-dependent algorithms (e.g. sorting)
    \item Photoshop filters
    \item Procedural generation of X in games (usually maps)
    \item Scientific simulations
\end{itemize}

%\begin{thebibliography}{9}
%
%    \bibitem{msvc-github}
%    Casey Carter
%    \emph{Issue optimizing the implementation of \texttt{uniform_int_distribution}}
%    \url{https://github.com/microsoft/STL/issues/178#issuecomment-553059724}
%
%\end{thebibliography}

\end{document}
