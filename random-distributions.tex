\documentclass{wg21}

\usepackage{xcolor}
\usepackage{soul}
\usepackage{ulem}
\usepackage{fullpage}
\usepackage{parskip}
\usepackage{csquotes}
\usepackage{relsize}      % provide relative font size changes
\usepackage{textcomp}     % provide \text{l,r}angle
\usepackage{mathrsfs}     % mathscr font
\usepackage{microtype}

\title{Make Pseudo-random Numbers Portable}
\docnumber{PXXXXR0}
\audience{LEWGI}
\author{Martin Hořeňovský}{martin.horenovsky@gmail.com}

\usepackage[inner=2cm,outer=2cm]{geometry}

\begin{document}
\maketitle

%\begin{flushright}
%\hfill \break
%\hfill \break
%\textit{Some tagline?}
%\end{flushright}

% URNGs are portable:
% http://eel.is/c++draft/rand#predef

\hypertarget{introduction}{%
	\section{Introduction}\label{introduction}}

The C++ standard library provides a set of utilities to generate
pseudo-random numbers in a range of user's choosing. This is done
by applying a type that represents the desired statistical distribution
(e.g. \texttt{uniform\_int\_distribution}) with a type conforming
to the Universal Random Bit Generator (URBG) concept (e.g. \texttt{mt19937}).

This paper proposes changes that would allow users to use these facilities
for projects that require cross-platform reproducibility, such as games.


\hypertarget{motivation}{%
    \section{Motivation}\label{motivation}}

A very simple ``Hello world'' of random number generation looks something
like this:

% https://godbolt.org/z/dB7H_2
\begin{codeblock}
#include <random>
#include <iostream>

int main() {
    std::mt19937 urbg;
    std::uniform_int_distribution<> dist(0, 100);
    std::cout << dist(urbg) << '\n';
}
\end{codeblock}

On my machine, this code always prints out \texttt{53}. \href{https://godbolt.org/z/dB7H_2}{On other
popular platforms, the code prints \texttt{82} and \texttt{92}}. This
difference in outputs makes the standard-provided \texttt{<random>}
facilities unusable in contexts where a cross-platform reproducibility
is required, such as procedural generation and physics simulations.

However, cross-platform reproducibility is useful even in cases
where it is not required, such as property-based tests. For property
based testing, it is desirable for a fixed seed to generate the same
test case across different platforms, so that developers can share
interesting seeds between themselves.

Because of this, \texttt{<random>} is currently unusable in many
domains and applications, and if we want to change that, we need
to ensure that a deterministically seeded code, such as the snippet
above, generates the same numbers on all platforms.


\hypertarget{possible-solutions}{%
    \section{Possible solutions}\label{possible-solutions}}

This section outlines possible solutions to the problem explained above,
together with their advantages and disadvantages as seen by the author of
this paper.

\hypertarget{standardize-implementation}{%
    \subsection{Standardize implementation for current distributions}\label{possible-solutions}}

One possibility (prefered by the author), is to mandate the specific
implementation of all distributions in \texttt{<random>}. This option
has two advantages in that it retroactively fixes broken code, and
also does not introduce further maintenance burden on standard library
implementations (unlike the later options).

The disadvantage of this option is that it can be viewed as a breaking
change, and that it takes away the implementation's latitude to provide
fastest possible distributions.

I see the backward compatibility argument as debatable, as at least one
implementor does not see the change to distribution output as breaking,
unless it forces ABI break\cite{msvc-github}.

The second disadvantage can be fixed by adding \texttt{fast_meow_distributions}
to the standard, at the cost of also erasing the maintenance burden advantage
of this option.


\hypertarget{portable-overloads}{%
    \subsection{Provide \texttt{portable_meow_distributions}}\label{portable-overloads}}

The other possibility is to add \texttt{portable_meow_distributions}
to \texttt{<random>}. This option does not break old code that relies
on platform's specific implementation, and let's implementors provide
the best optimized code for distributions for their platforms.

However, it comes at the cost of increased maintenance burden (compared
to just standardizing implementations of plain distributions), being
harder to teach, and keeping existing single-platform code broken if it
becomes used in cross-platform contexts.



\hypertarget{non-goals}{%
    \section{Non-goals}\label{non-goals}}

TODO:

* No attempts to fix other <random> problems in this paper...

* Provide the code for distributions -- future paper if there is interest in this direction

\vskip1cm
\hrule

\begin{thebibliography}{9}
	
	\bibitem{msvc-github}
    Casey Carter
    \emph{Issue optimizing the implementation of \texttt{uniform_int_distribution}}
    \url{https://github.com/microsoft/STL/issues/178#issuecomment-553059724}
    	
\end{thebibliography}

\end{document}
